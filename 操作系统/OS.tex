\documentclass{ctexart}
\usepackage[colorlinks=true,linkcolor=black]{hyperref}
\usepackage{listings}

\lstset{
    language=C,                      % 代码语言
    keywordstyle=\color{blue},       % 关键词颜色
    numbers=left,                    % 显示行号
    numberstyle=\tiny,               % 行号样式
    stepnumber=1                     % 行号递增步长
}


\title{操作系统}
\author{陶}
\date{\today}

\begin{document}

\maketitle
\tableofcontents

\section{操作系统的介绍}
\label{sec:first}
\subsection{
    当程序运行时发生了什么
}
执行指令:处理器从memory fetch instructions,解码并且执行。(the basic of Von Neumann model of computing)

\subsection{操作系统的各种称呼(目标都是为了更好用)}
虚拟机:将处理器,内存,磁盘等虚拟化
标准库:提供了各种api方便系统调用
资源管理器:管理处理器,内存,磁盘等资源
\subsection{三个主题}
\subsubsection{Persistence}
内存断电后清空,但用户不希望丢掉数据:使用file system管理文件,以及发展出I/O设备(HD SSD)长期储存
\subsubsection{Concurrency}
\subsubsection{Virtualization}

\subsection{设备驱动}
操作系统下载驱动后明白如何使用该设备
\subsection{写的协议}
\subsubsection{journaling}
\subsubsection{copy-on-write}
\subsection{如何构建系统}
\subsection{目的}
\subsubsection{performance}
\subsubsection{protection}
isolation
\subsubsection{reliability}
\subsubsection{energy-efficiency}
\subsubsection{security}
\subsubsection{mobility}
\subsection{Abstraction}



\section{The Abstraction: The Process----running program}
\label{sec:second}
虚拟化CPU
\subsection{policies}
algorithms for making some kind of decision within the OS

\subsection{进程的组成}
\subsubsection{进程映像}
程序,数据集合,栈,PCB(process control block)
构成进程在系统中存在和活动的实体
\subsubsection{PCB:进程描述块,进程存在的唯一标志}
\label{pcb}
进程名:标识对应进程的唯一标识符或数字

特征信息:是系统进程还是用户进程,进程实体是否常驻内存

进程状态信息:运行态/就绪态/阻塞态

调度优先权:进程获取CPU的优先级别

通信信息:反映该进程与哪些进程有什么样的通信关系,如等待哪个进程的信号。

现场保护区:对应进程由于某个原因放弃使用CPU时,需要把它的一部分与运行环境有关的信息保存起来,以便在重新获得CPU后可以恢复正常运行。
通常被保护的信息有程序计数器,程序状态字,工作寄存器的内容等。

资源需求,分配和控制方面的信息:如进程所需或占用的I/O设备,磁盘空间,数据区等

进程实体信息:指出该进程的程序和数据的存储情况,在内存或外存的地址,大小等。

族系关系:反映父子进程的隶属关系

其他信息:文件信息,工作单元

\subsection{进程队列}
为了对进程进行有效管理,将各进程的PCB组织起来形成进程队列。
\subsubsection{线性方式}
数组。操作系统预先确定整个系统中同时存在的进程最大数目,静态分配空间。
\subsubsection{链接方式}
按照进程的不同状态,将其分别放在不同链表中。阻塞队列可以有多个对应不同的阻塞原因。
实际系统中就绪队列按照进程优先级分成多个队列,有同一优先级的排在同一个队列上。
\subsubsection{索引方式}
根据进程不同的状态建立索引表,索引表条目存放PCB地址,将索引表的起始地址放在专用的指针单元中
\subsection{进程管理}
\subsubsection{创建进程 fork()}

1.从系统的PCB表中找出一个空闲的PCB项,并指定唯一的进程标识号(Process identifier,PID),用作进程内部名

2.根据调用者提供的所需内存大小,为新进程分配必要的内存空间,用于存放其程序,数据和工作区。
有两种可能:
(1)子进程复制父进程的地址空间
(2)将新的程序装入子进程的地址空间

3.根据调用者提供的参数,将新进程的PCB初始化。参数包括新进程名(外部标识符),父进程标识符,处理机初始状态,进程优先级,本进程的开始地址等。一般将新进程状态设置为就绪态。

4.一个新进程派生新进程后,有两种可能的执行方式:
(1)异步方式:父进程继续运行,子进程也可以被调度运行
(2)同步方式:父进程睡眠,等待某个或全部子进程终止,然后继续运行

父进程调用fork创建子进程时,将自己的地址空间制作一个副本,其中包括User结构,正文段,数据段,用户栈和系统栈,使得父进程很容易和子进程通信。两个进程都可以继续执行fork后的指令。
当fork的返回值(子进程PID)不等于0,表示父进程在执行,当fork 返回值为0时表示子进程在执行

此时子进程中的程序为父进程调用fork后的指令。

subsubsection{终止进程 primitive}

1.从系统的PCB表中找到指定进程的PCB。若它处于运行态,则立即终止该程序的运行。

2.回收该进程所占用的全部资源

3.如果该程序有子孙进程,则要终止其所有子孙进程并且回收它们的全部资源

4.释放原本进程的PCB,并将其从原本队列中摘下

\subsubsection{阻塞进程 sleep}

不满足继续的条件,主动调用sleep阻塞自己。

1.立即停止当前进程的执行

2.将该进程的CPU现场送到该进程的PCB现场保护区中保存

3.将该进程中PCB的现行状态由运行态改为阻塞态,并将其插入有相同事件的阻塞队列中。

4.转到进程调度程序,重新从就绪队列中挑选一个合适的进程运行

\subsubsection{唤醒进程 wakeup}

当阻塞进程所等待的事件出现时(等待数据到达或I/O操作完成)

1.将被阻塞进程从阻塞队列摘下

2.将现行状态改为就绪态,将该进程插入就绪队列

3.如果被唤醒的进程比正在运行的进程有更高的优先级,则重新调度标志

\textbf{sleep 为自己主动沉睡,wakeup 为将别人唤醒}

\subsubsection{更换进程映像}

1.释放子进程原来的程序和数据所占用的内存空间

2.从磁盘上找出子进程所要执行的程序和数据(通常以可执行文件的形式存放————ELF Executable and Linkable Format)

3.分配内存空间,装入新的程序和数据

4.为子进程建立初始运行环境:对各个寄存器初始化,使其返回用户态,进行该进程的程序

\textbf{不同的操作系统有不同的实现方式}

\section{Linux 进程管理}
\subsection{linux 进程状态}

1.运行态(TASK\_RUNNING):运行+就绪态。当前进程由运行指针所指向。

2.可中断等待态(TASK\_INTERRUPTIBLE):“浅度”睡眠,能被信号,中断或所等待资源被满足时唤醒。

3.不可中断等待态(TASK\_UNINTERRUPTIBLE):“深度”睡眠,只能在所等待资源被满足时唤醒。

4.停止态(TASK\_STOPPED):通常由于接收一个信号致使进程停止,正在被调试的进程可能处于停止态。

5.僵死态(TASK\_ZOMBIE):由于某些原因,程序被终止了,但该进程的控制结构task\_struct仍然保留着。

\subsection{linux 进程模式}
用户模式(用户态)和内核模式(核心态)
\subsubsection{用户态:当前运行的是用户程序,应用程序,或者内核之外的系统程序}

\subsubsection{核心态:用户态时出现系统调用或者发生中断事件(运行操作系统(核心)程序)}
可以执行机器的特权指令,不受用户的干预(即使是root用户)。

\vspace{\baselineskip}

根据进程的功能和运行的程序,可以将进程分为两大类
\subsubsection{系统进程}
只在核心态运行,执行操作系统的代码,完成管理性的工作,如内存分配和进程切换。
\subsubsection{用户进程}
既可以在用户态下运行,也可以在核心态下运行。

当用户进程需要进行一些需要特权的操作时(例如访问硬件、执行特权指令等),它会通过系统调用进入内核态,请求操作系统执行相关的特权操作。操作系统会在内核态中执行相应的任务,然后将控制返回给用户态。
\subsection{linux 进程的结构}
每个进程都有一个名为task\_struct的数据结构,相当于PCB。

系统中有一个名为task的向量数组,长度默认为521B。

创建新进程时,Linux从系统内存中分配一个task\_struct结构,并将它的首地址放入task中,当前运行进程的task\_struct由current指针指示。

\subsubsection{task\_struct}
与之前\hyperref[pcb]{PCB}相比额外的信息

时间和计时器:内核要记录进程的创建时间和进程运行所用的CPU时间。linux系统支持进程的时间间隔计时器。

文件系统:进程在运行时可以打开和关闭文件。task\_struct结构中包含指向每个打开文件的文件描述符的指针,并且由两个指向虚拟文件系统(virtual file system VFS)索引节点的指针。第一个索引节点是进程的根目录。第二个索引节点是当前的工作目录。

两个索引节点都有一个计数字段,该计数字段记录访问该索引节点的进程数。

虚拟内存: linux系统必须了解如何将虚拟内存映射到系统的物理内存

% 处理器信息:每个进程运行时都要使用处理器的寄存器及堆栈等资源,当一个进程挂起时,所有有关处理器的内容都要保存到进程的task\_struct中。当进程恢复运行时,所有保存的内容再装回处理器中。

\subsubsection{系统堆栈}
保存中断现场信息和进程进入核心态后执行子程序(函数)嵌套调用的返回现场信息。(在没中断的时候,它存好寄存器和函数调用的情况,中断发生后它往栈多Push一条执行到第几行的信息,在之后继续执行)

因为系统堆栈和task\_struct结构存在紧密联系,两者的物理储存空间也连在一起。

内核在为每个进程分配task\_struct结构的内存空间时,实际上一次分配两个连续的内存页面(8KB),底部约1KB的空间用于存放task\_struct结构,上面约7KB的空间存放进程的系统堆栈。

另外,系统空间堆栈的大小是静态确定的,而用户空间堆栈可以在运行时动态扩展。

\section{操作命令}
\subsection{有关进程}
\subsubsection{ps 查看进程状态}
PID:进程标识号

TTY:该进程建立时所对应的终端,“?”表示该进程不占用终端

TIME:进程累计使用CPU的时间,有些运转很长时间的命令使用CPU的时间很少,所以该字段往往是00:00:00

CMD:执行进程的命令名

\vspace{\baselineskip}

-a 显示系统中与TTY相关的(会话组长除外,因为不是与会话有关的)所有进程的信息

-e 显示所有进程的信息

-f 显示进程的所有信息

UID:进程属主的用户ID号

PPID:父进程的ID号

C:进程最近使用CPU的估算

STIME:进程开始时间,格式为 小时:分

-l 以长格式显示进程信息

-r 只显示正在运行的进程

-u 显示面向用户的格式(包括用户名,CPU及内存使用情况,进程运行态)

-x 显示所有终端上的进程信息

\subsection{kill}
信号机制(也被称为软中断)是在软件层次上对中断机制的一种模拟。异步进程可以通过彼此发送信号来实现简单通信。

系统预先规定若干个不同类型的信号(x86中Linux内核设置了32种信号,现在的linux和posix.4定义了64种信号)

当进程遇到特定事件或出现特定要求时,就把一个信号写到相应进程task\_struct的signal位。

接收信号的进程在运行过程中要检测自身是否收到了信号,如果已收到信号,则转去执行预先规定好的信号处理程序。处理之后,再返回原先正在执行的进程。

可以通过ctrl+c来终止进程,也可以用kill.但对于一个后台进程只能用kill来终止。

-s:指定要发送的信号,可以是信号名也可以是对应信号的编号

-p:显示进程所属的进程组号

-l:显示信号名列表,这也可以在/usr/include/linux/signal.h文件中找到

\textbf{1.使用kill时如果没带信号会默认使用编号15信号杀死程序}

\textbf{2.撤销时kill必须是有足够的权限,撤销一个不存在的或者无权限的进程将会报错}

\textbf{3.终端会显示信息,有时要按enter后才显示}

\textbf{4.可以输入多个pid}

\textbf{5.可以使用kill 0 来撤销当前shell运行的所有进程,省去搜索进程号的麻烦}

\subsubsection{sleep}
sleep 100 单位为秒
\subsubsection{fork}
\textbf{pid\_t}为有符号整型量
在父进程中返回子进程的PID(>0),在子进程中返回0,失败则返回-1

\subsubsection{exec}

execve才是真正执行的函数,其他都是包装过的库函数。作用为更换进程映像,根据指定文件名找到可执行文件,并用它来取代调用进程的内容。

argv和envp分别是传给被执行文件的命令行参数数组和环境变量数组

arg:命令行单个参数

path:表示完整路径

file:自动在环境变量寻找该文件

\subsubsection{wait}
wait(int* status)等待任何要僵死的进程,有关子进程退出时的一些状态保存在参数status中。若成功返回该终止进程的PID,否则返回-1.

如果出现没有子进程需要等待退出、调用wait的进程没有子进程等待退出等情况,wait函数会返回-1并设置errno为一个对应的错误值,可以通过检查errno来获取更具体的错误信息。

有多个子进程同时退出(如子进程很多),wait函数将会返回任意一个已经退出的子进程的PID。这种情况下,父进程可以通过循环调用wait函数,直到所有子进程都被处理完毕。

因为wait函数只能等待一个子进程的退出,所以在处理多个子进程退出的情况下,父进程可能需要在循环中反复调用wait函数。每次调用wait函数都会返回一个已经退出的子进程的PID,直到全部子进程都被处理完毕为止。

waitpid(pid\_t pid, int* status, int option); 

pid:

等待任意子进程: 如果将 pid 设置为 -1,表示等待任意子进程退出。这在父进程不关心具体子进程是哪个时非常有用。

等待同一进程组的任意子进程: 如果将 pid 设置为 0,表示等待与调用进程在同一进程组的任何子进程。这对于处理一组相关的子进程很有用。

等待指定进程ID的子进程: 如果将 pid 设置为具体的进程ID,表示等待具有该进程ID的子进程退出。这允许父进程有选择地等待特定的子进程。

option规定了该调用的行为:

WNOHANG (1): 如果指定了这个选项,并且没有子进程处于需要等待的状态,waitpid 将立即返回0,而不会阻塞父进程。

WUNTRACED (2): 此选项用于报告已经停止的子进程。如果子进程被暂停(例如,收到了SIGSTOP信号),但尚未终止,则返回其pid,将父进程status设置为相应(如阻塞)状态。

WCONTINUED (4): 此选项用于报告已经继续运行的子进程。如果子进程之前被停止,现在又继续运行了,waitpid 将返回其pid, 将父进程status设置为相应状态。

可以对它们执行逻辑‘或’

\begin{lstlisting}
    if (WIFEXITED(status)) {
        // Child process exited normally
        int exit_status = WEXITSTATUS(status);
        // Handle exit status
    } else if (WIFSIGNALED(status)) {
        // Child process terminated due to a signal
        int signal_number = WTERMSIG(status);
        // Handle termination signal
    } else if (WIFSTOPPED(status)) {
        // Child process has been stopped
        int stop_signal = WSTOPSIG(status);
        // Handle stop signal
    } else if (WIFCONTINUED(status)) {
        // Child process resumed execution
        // Handle the case of resumption
    }
\end{lstlisting}

\subsubsection{exit}
status:进程退出时的状态

\_exit函数比exit简单,仅仅是终止进程

exit需要检查文件的打开情况,清理I/O缓存才退出
\subsubsection{getpid}
返回该进程pid
\subsubsection{getppid}
返回父进程pid
\subsubsection{nice}

\section{系统调用的使用方式}
在linux/unix系统中,系统调用和库函数都是以C函数的形式提供给用户的,头文件一般放在/usr/include/sys或者/usr/include/linux目录下。

虽然系统调用类似库函数调用,但库函数属于用户层,只能在用户态运行不能进入核心态。系统调用属于操作系统的核心层,在核心态运行,而且可以实现处理机从用户态到核心态的转变。

\begin{lstlisting}
    #include <unistd.h>
    #include <sys/types.h>
    #include <stdio.h>
    int main(int argc, char *argv[]){
        int pid;
        pid = fork();
        if (pid < 0) {
            fprintf(stderr, "Fork failed");
            exit(1);
        }
        else if (pid == 0) {
            execlp ("/bin/ls", "ls", NULL); //execlp (const char *file, const char *arg, ... )
        }
        else {
            wait(NULL);
            printf("Child Complete");
            exit(0);
        }
    }
\end{lstlisting}

除了0外,任何进程都必须有父进程,如果父进程先于子进程死亡或退出,则子进程会被指定一个新的父进程init(其PID为1)
\clearpage
\section{竞争}
\subsection{竞争条件}
假定进程P\textsubscript{a}负责为用户分配打印机,P\textsubscript{b}负责释放打印机。系统中设立一个打印机分配表,由各个进程共用。
\begin{table}[h]  % 放置表格的位置,可选参数[h]表示“here”
    \centering     % 居中
    \caption{打印机分配表(初始情况)}  % 表格标题
    \begin{tabular}{|c|c|c|c|}  % 列的格式:居中对齐(c),竖线表示纵向边框
        \hline
        \textbf{打印机编号} & \textbf{分配标志} & \textbf{用户名} & \textbf{用户定义的设备名} \\  % 表头
        \hline
        0 & 1 & Meng & PRINT \\  % 表格内容
        \hline
        1 & 0 &  &\\
        \hline
        2 & 1 & Liu & OUTPUT \\
        \hline
    \end{tabular}
\end{table}



P\textsubscript{a}进程分配打印机的过程:

1.逐项检查分配标志,找出标志为0的打印机号码

2.把该打印机的分配标志置为1

3.把用户名和设备名填入分配表中相应位置

P\textsubscript{b}进程释放打印机的过程:

1.逐项检查分配表的各项信息,找出分配标志为1且用户名设备名与被释放的名字相同的打印机编号。

2.将该打印机的分配标志置位0

3.清除该打印机的用户名和设备名

如果P\textsubscript{a} P\textsubscript{b}并行,以以下顺序执行:

P\textsubscript{b}:

1.查分配表,找到分配标志为1,用户名为MENG,设备名为PRINT的打印机。(为0号打印机)

2.将0号打印机的分配标志置为0

P\textsubscript{a}:

1.逐项检查分配标志,找出标志为0的打印机号码(0号打印机)

2.把该打印机的分配标志置为1

3.把用户名Zhang和设备名LP填入分配表中相应位置

P\textsubscript{b}:

3.清除0号打印机的用户名和设备名

结果:
\begin{table}[h]  % 放置表格的位置,可选参数[h]表示“here”
    \centering     % 居中
    \caption{打印机分配表(出错情况)}  % 表格标题
    \begin{tabular}{|c|c|c|c|}  % 列的格式:居中对齐(c),竖线表示纵向边框
        \hline
        \textbf{打印机编号} & \textbf{分配标志} & \textbf{用户名} & \textbf{用户定义的设备名} \\  % 表头
        \hline
        0 & 1 &  &  \\  % 表格内容
        \hline
        1 & 0 &  &\\
        \hline
        2 & 1 & Liu & OUTPUT \\
        \hline
    \end{tabular}
\end{table}
最后0号机无法被释放,也无法被再次使用。

当两个进程同时访问和操纵相同的数据时,最后的执行结果取决于进程运行的精确时序,这种情况被称为\textbf{竞争条件}(race condition)

\subsection{临界资源 critical resource}
打印机,读卡机,公共变量,表格等资源都是临界资源,也就是独占型资源。
\subsection{临界区 critical section CS}
每个进程访问临界区的那段程序叫临界区。
\section{进程通信}
\subsection{同步}
在上一道工序未完成或加工质量不合格时,下一道工序不能继续下去。
\subsection{互斥}
汽车在岔口争用车道

实现机制:

\textbf{软件:}

原语操作

软件锁

\textbf{硬件:}

\subsubsection{禁止中断}

使每个进程进入临界区后立即关闭所有的中断,在它即将离开临界区之前才开放中断。

由于禁止中断,时钟中断也被禁止,不会把CPU切换到其它进程。

这种把关闭中断的权利交给用户进程的方法,一旦某个进程关闭中断后,如果它不再开放中断,则系统可能会因此中止。

\subsubsection{设置专用机器指令}



\clearpage

\section*{导航}
\label{sec:navigation}

在这里,你可以添加一些导航链接,如链接到、\hyperref[subsec:sub]{子节}、\hyperref[sec:second]{第二节}等。

\end{document}
