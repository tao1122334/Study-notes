\documentclass{ctexart}
\usepackage[colorlinks=true,linkcolor=black]{hyperref}

\title{操作系统}
\author{陶}
\date{\today}

\begin{document}

\maketitle
\tableofcontents

\section{操作系统的介绍}
\label{sec:first}
\subsection{
    当程序运行时发生了什么
}
执行指令:处理器从memory fetch instructions,解码并且执行。(the basic of Von Neumann model of computing)

\subsection{操作系统的各种称呼(目标都是为了更好用)}
虚拟机:将处理器,内存,磁盘等虚拟化
标准库:提供了各种api方便系统调用
资源管理器:管理处理器,内存,磁盘等资源
\subsection{三个主题}
\subsubsection{Persistence}
内存断电后清空,但用户不希望丢掉数据:使用file system管理文件,以及发展出I/O设备(HD SSD)长期储存
\subsubsection{Concurrency}
\subsubsection{Virtualization}

\subsection{设备驱动}
操作系统下载驱动后明白如何使用该设备
\subsection{写的协议}
\subsubsection{journaling}
\subsubsection{copy-on-write}
\subsection{如何构建系统}
\subsection{目的}
\subsubsection{performance}
\subsubsection{protection}
isolation
\subsubsection{reliability}
\subsubsection{energy-efficiency}
\subsubsection{security}
\subsubsection{mobility}
\subsection{Abstraction}



\section{The Abstraction: The Process----running program}
\label{sec:second}
虚拟化CPU
\subsection{policies}
algorithms for making some kind of decision within the OS

\subsection{进程的组成}
\subsubsection{进程映像}
程序,数据集合,栈,PCB(process control block)
构成进程在系统中存在和活动的实体
\subsubsection{PCB:进程描述块,进程存在的唯一标志}
\label{pcb}
进程名:标识对应进程的唯一标识符或数字

特征信息:是系统进程还是用户进程,进程实体是否常驻内存

进程状态信息:运行态/就绪态/阻塞态

调度优先权:进程获取CPU的优先级别

通信信息:反映该进程与哪些进程有什么样的通信关系,如等待哪个进程的信号。

现场保护区:对应进程由于某个原因放弃使用CPU时,需要把它的一部分与运行环境有关的信息保存起来,以便在重新获得CPU后可以恢复正常运行。
通常被保护的信息有程序计数器,程序状态字,工作寄存器的内容等。

资源需求,分配和控制方面的信息:如进程所需或占用的I/O设备,磁盘空间,数据区等

进程实体信息:指出该进程的程序和数据的存储情况,在内存或外存的地址,大小等。

族系关系:反映父子进程的隶属关系

其他信息:文件信息,工作单元

\subsection{进程队列}
为了对进程进行有效管理,将各进程的PCB组织起来形成进程队列。
\subsubsection{线性方式}
数组。操作系统预先确定整个系统中同时存在的进程最大数目,静态分配空间。
\subsubsection{链接方式}
按照进程的不同状态,将其分别放在不同链表中。阻塞队列可以有多个对应不同的阻塞原因。
实际系统中就绪队列按照进程优先级分成多个队列,有同一优先级的排在同一个队列上。
\subsubsection{索引方式}
根据进程不同的状态建立索引表,索引表条目存放PCB地址,将索引表的起始地址放在专用的指针单元中
\subsection{进程管理}
\subsubsection{创建进程 fork()}

1.从系统的PCB表中找出一个空闲的PCB项,并指定唯一的进程标识号(Process identifier,PID),用作进程内部名

2.根据调用者提供的所需内存大小,为新进程分配必要的内存空间,用于存放其程序,数据和工作区。
有两种可能:
(1)子进程复制父进程的地址空间
(2)将新的程序装入子进程的地址空间

3.根据调用者提供的参数,将新进程的PCB初始化。参数包括新进程名(外部标识符),父进程标识符,处理机初始状态,进程优先级,本进程的开始地址等。一般将新进程状态设置为就绪态。

4.一个新进程派生新进程后,有两种可能的执行方式:
(1)异步方式:父进程继续运行,子进程也可以被调度运行
(2)同步方式:父进程睡眠,等待某个或全部子进程终止,然后继续运行

父进程调用fork创建子进程时,将自己的地址空间制作一个副本,其中包括User结构,正文段,数据段,用户栈和系统栈,使得父进程很容易和子进程通信。两个进程都可以继续执行fork后的指令。
当fork的返回值(子进程PID)不等于0,表示父进程在执行,当fork 返回值为0时表示子进程在执行

此时子进程中的程序为父进程调用fork后的指令。

subsubsection{终止进程 primitive}

1.从系统的PCB表中找到指定进程的PCB。若它处于运行态,则立即终止该程序的运行。

2.回收该进程所占用的全部资源

3.如果该程序有子孙进程,则要终止其所有子孙进程并且回收它们的全部资源

4.释放原本进程的PCB,并将其从原本队列中摘下

\subsubsection{阻塞进程 sleep}

不满足继续的条件,主动调用sleep阻塞自己。

1.立即停止当前进程的执行

2.将该进程的CPU现场送到该进程的PCB现场保护区中保存

3.将该进程中PCB的现行状态由运行态改为阻塞态,并将其插入有相同事件的阻塞队列中。

4.转到进程调度程序,重新从就绪队列中挑选一个合适的进程运行

\subsubsection{唤醒进程 wakeup}

当阻塞进程所等待的事件出现时(等待数据到达或I/O操作完成)

1.将被阻塞进程从阻塞队列摘下

2.将现行状态改为就绪态,将该进程插入就绪队列

3.如果被唤醒的进程比正在运行的进程有更高的优先级,则重新调度标志

\textbf{sleep 为自己主动沉睡,wakeup 为将别人唤醒}

\subsubsection{更换进程映像}

1.释放子进程原来的程序和数据所占用的内存空间

2.从磁盘上找出子进程所要执行的程序和数据(通常以可执行文件的形式存放————ELF Executable and Linkable Format)

3.分配内存空间,装入新的程序和数据

4.为子进程建立初始运行环境:对各个寄存器初始化,使其返回用户态,进行该进程的程序

\textbf{不同的操作系统有不同的实现方式}

\section{Linux 进程管理}
\subsection{linux 进程状态}

1.运行态(TASK\_RUNNING):运行+就绪态。当前进程由运行指针所指向。

2.可中断等待态(TASK\_INTERRUPTIBLE):“浅度”睡眠,能被信号,中断或所等待资源被满足时唤醒。

3.不可中断等待态(TASK\_UNINTERRUPTIBLE):“深度”睡眠,只能在所等待资源被满足时唤醒。

4.停止态(TASK\_STOPPED):通常由于接收一个信号致使进程停止,正在被调试的进程可能处于停止态。

5.僵死态(TASK\_ZOMBIE):由于某些原因,程序被终止了,但该进程的控制结构task\_struct仍然保留着。

\subsection{linux 进程模式}
用户模式(用户态)和内核模式(核心态)
\subsubsection{用户态:当前运行的是用户程序,应用程序,或者内核之外的系统程序}

\subsubsection{核心态:用户态时出现系统调用或者发生中断事件(运行操作系统(核心)程序)}
可以执行机器的特权指令,不受用户的干预(即使是root用户)。

\vspace{\baselineskip}

根据进程的功能和运行的程序,可以将进程分为两大类
\subsubsection{系统进程}
只在核心态运行,执行操作系统的代码,完成管理性的工作,如内存分配和进程切换。
\subsubsection{用户进程}
既可以在用户态下运行,也可以在核心态下运行。

当用户进程需要进行一些需要特权的操作时(例如访问硬件、执行特权指令等),它会通过系统调用进入内核态,请求操作系统执行相关的特权操作。操作系统会在内核态中执行相应的任务,然后将控制返回给用户态。
\subsection{linux 进程的结构}
每个进程都有一个名为task\_struct的数据结构,相当于PCB。

系统中有一个名为task的向量数组,长度默认为521B。

创建新进程时,Linux从系统内存中分配一个task\_struct结构,并将它的首地址放入task中,当前运行进程的task\_struct由current指针指示。

\subsubsection{task\_struct}
与之前\hyperref[pcb]{PCB}相比额外的信息

时间和计时器:内核要记录进程的创建时间和进程运行所用的CPU时间。linux系统支持进程的时间间隔计时器。

文件系统:进程在运行时可以打开和关闭文件。task\_struct结构中包含指向每个打开文件的文件描述符的指针,并且由两个指向虚拟文件系统(virtual file system VFS)索引节点的指针。第一个索引节点是进程的根目录。第二个索引节点是当前的工作目录。

两个索引节点都有一个计数字段,该计数字段记录访问该索引节点的进程数。

虚拟内存: linux系统必须了解如何将虚拟内存映射到系统的物理内存

% 处理器信息:每个进程运行时都要使用处理器的寄存器及堆栈等资源,当一个进程挂起时,所有有关处理器的内容都要保存到进程的task\_struct中。当进程恢复运行时,所有保存的内容再装回处理器中。

\subsubsection{系统堆栈}
保存中断现场信息和进程进入核心态后执行子程序(函数)嵌套调用的返回现场信息。(在没中断的时候,它存好寄存器和函数调用的情况,中断发生后它往栈多Push一条执行到第几行的信息,在之后继续执行)

因为系统堆栈和task\_struct结构存在紧密联系,两者的物理储存空间也连在一起。

内核在为每个进程分配task\_struct结构的内存空间时,实际上一次分配两个连续的内存页面(8KB),底部约1KB的空间用于存放task\_struct结构,上面约7KB的空间存放进程的系统堆栈。

另外,系统空间堆栈的大小是静态确定的,而用户空间堆栈可以在运行时动态扩展。

\section{操作命令}
\subsection{有关线程}
\subsubsection{ps 查看进程状态}
PID:进程标识号

TTY:该进程建立时所对应的终端,“?”表示该进程不占用终端

TIME:进程累计使用CPU的时间,有些运转很长时间的命令使用CPU的时间很少,所以该字段往往是00:00:00

CMD:执行进程的命令名

\vspace{\baselineskip}

-a 显示系统中与TTY相关的(会话组长除外,因为不是与会话有关的)所有进程的信息

-e 显示所有进程的信息

-f 显示进程的所有信息

UID:进程属主的用户ID号

PPID:父进程的ID号

C:进程最近使用CPU的估算

STIME:进程开始时间,格式为 小时:分

-l 以长格式显示进程信息

-r 只显示正在运行的进程

-u 显示面向用户的格式(包括用户名,CPU及内存使用情况,进程运行态)

-x 显示所有终端上的进程信息

\subsection{kill}
信号机制(也被称为软中断)是在软件层次上对中断机制的一种模拟。异步进程可以通过彼此发送信号来实现简单通信。

系统预先规定若干个不同类型的信号(x86中Linux内核设置了32种信号,现在的linux和posix.4定义了64种信号)

当进程遇到特定事件或出现特定要求时,就把一个信号写到相应进程task\_struct的signal位。

接收信号的进程在运行过程中要检测自身是否收到了信号,如果已收到信号,则转去执行预先规定好的信号处理程序。处理之后,再返回原先正在执行的进程。

可以通过ctrl+c来终止进程,也可以用kill.但对于一个后台进程只能用kill来终止。


\clearpage

\section*{导航}
\label{sec:navigation}

在这里,你可以添加一些导航链接,如链接到、\hyperref[subsec:sub]{子节}、\hyperref[sec:second]{第二节}等。

\end{document}
