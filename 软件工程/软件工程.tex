\documentclass{ctexart}
\usepackage[colorlinks=true,linkcolor=black]{hyperref}

\title{软件工程}
\author{陶}
\date{\today}

\begin{document}

\maketitle
\tableofcontents

\section{requirement}
\label{sec:first}
需求(requirement)描述了从客户的角度,这个系统要做什么,这个系统要提供什么服务以及它操作期间的限制。

\subsection{functionality}
\subsection{constrains}
\subsection{goals}
\subsection{为什么一个项目会失败}
50\%:不足的需求定义

15\%:不足的范畴定义

17\%: 不足的风险管理

14\%: 沟通问题

3\%: 缺少有效资源

1\%: 其它
\subsection{需求的重要性}
越晚明白需求的定义,成本就越高
\subsection{需求来自哪里}
应用领域:如银行的法律和规定
Stakeholders:利害关系人
\subsection{primary Stakeholders}
Customers:决定主要需求和项目范畴以及与主要负责人签订合同

Project Manager: 控制整个项目的生产过程,考虑所有利益相关者的利益和需求,按时用预算创建有力产品,
监管生产过程,做必要的协调。

Business Analysts: 一个商业分析者分析客户的想法,和开发团队沟通并且确定项目范畴和需求。

Development Team: 负责按时交付和测试软件

Quality assurance: 创建并且运行测试,找出bug,并且提供反馈

UI\&UX designers: 让产品的交互界面对用户友好并且易懂,让客户很快且容易地满足需求

End-users: 用户的需求影响系统的设计和功能。使用者可以被聚集起来作为测试产品的第二道,提供最初始的反馈。他们可以指出产品缺少的功能并且给用户体验做出贡献。

Government:采用有监管力的国际标准,对不遵守规定的软件罚款。

Competitors: 研发新功能并且影响市场趋势,带来新挑战。

\subsection{需求的种类}
从高到低:

商业角度:主要用途,为什么它被需要,以及它的使用范畴,可以得到什么商业效益,感兴趣的用户

愿景(vision)

用户角度:使用样例,用户场景,用户故事

系统需求:分为功能性和非功能性需求

例:

便利校园

方便身份检验  方便食堂就餐  方便借书

方便食堂就餐:方便充值 方便查询余额以及消费历史

方便借书: 有借书机器
\subsubsection{功能性需求}
功能性需求是系统应该提供的最基本功能

代表了系统输入,应有的操作和所需要的输出

用户可以直接看到的产品

例:

系统应提供的服务:按照关键字检索图书

针对特定输入的相应:对于格式不正确的身份证号提示并请重新输入

在特定情形下的行为: 用户5分钟无操作自动锁屏

不应做什么:不允许尝试密码输入三次以上

\subsubsection{non-functional requirement}
非功能性需求表示与系统功能无关的部分

决定了软件的质量属性

决定了系统的行为和基本特征以及影响用户体验的特点

例:

性能(performance): 联机刷卡应该在5s内返回结果

可靠性(reliability):系统整体可靠性要达到99.99\%以上

安全性(securi):系统应确保手机支付充值账户和密码不会被泄露和盗用

易用性(usability):用户根据提示学会手机支付充值时间不超过10分钟

产品约束(product constrains):软件系统要在已有的几台服务器上运行并使用Linux系统

过程约束(process constrains): 软件系统应该在5个月内交付并严格遵循给定的过程规范

\subsection{requirement analysis in scrum}
\subsubsection{user story example}
as a

registered user

i want to 

change my password

so i can 

keep my account secure

\clearpage

\section*{导航}
\label{sec:navigation}

在这里,你可以添加一些导航链接,如链接到、\hyperref[subsec:sub]{子节}、\hyperref[sec:second]{第二节}等。

\end{document}
