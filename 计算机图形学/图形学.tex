\documentclass{ctexart}
\usepackage[colorlinks=true,linkcolor=black]{hyperref}

\title{计算机图形学}
\author{陶}
\date{\today}

\begin{document}

\maketitle
\tableofcontents

\section{画一个三角形发生了什么}
\label{sec:first}
\subsection{实体和物理设备的选择}
应用的启动和API的使用通过VkInstance完成。

在创建好实体后可以查询Vulkan支持的硬件并且选择一或多的VkPhysicalDevices(比如针对VRAM和是否使用显卡)

\subsection{逻辑设备和排队族}
选好用什么硬件后需要构建VkDevice(logical device),来选择更多特性,比如多视图渲染和64比特浮点数。

你也需要确定


\subsection{window 窗口和转换链}
\subsection{图像查看和结构缓冲}
\subsection{渲染管线}
\subsection{图像管道}
\subsection{命令池和命令缓冲}
\subsection{主循环}


\clearpage

\section*{导航}
\label{sec:navigation}

在这里,你可以添加一些导航链接,如链接到、\hyperref[subsec:sub]{子节}、\hyperref[sec:second]{第二节}等。

\end{document}
