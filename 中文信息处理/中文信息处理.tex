\documentclass{ctexart}
\usepackage[colorlinks=true,linkcolor=black]{hyperref}

\title{中文信息处理}
\author{陶}
\date{\today}

\begin{document}

\maketitle
\tableofcontents

\section{规则派Vs统计派}
\label{sec:first}
\subsection{
    普通话中,“二”与“两”的区别?如何让机器正确填写以下片段?
}
规则派:

rule1:基数末尾、十位用二,不用两。十二、二十三、*十两、*两十五。

rule2:基数词百位以上,都可以用。

rule3:单个序数,同rule1。第二、*第两、初二、*初两。

rule4:数量名结构,度量衡以外,用“两”,不用“二”。两张桌子、*二张桌子。

rule5:度量衡做量词时,可以用“两”,也可以用“二”。两米、二米

优点:

如果规则完备,可生成所有合格的片段,避免所有不合格的片段。

缺点:

需要语言学专业人士参与总结。

统计派:

观察“二”、“两”在所有已知文本中的搭配,然后照抄。

优点:无需语言学专业人士参与。让机器做匹配即可。

缺点:从理论上说,出错的可能性总是有的。

\vspace{\baselineskip}

\textbf{互联网带来的大量数据是统计派占上风的根源}

\section{文本的处理}
\subsection{语言学单位角度}
\subsubsection{汉字}
汉字编码

字库建设

汉字输入

汉字显示

\subsubsection{词}

中文分词

词库建设

词性标注

命名实体识别

\subsubsection{句子}

句法成分

论元结构

配价

语义特征

歧义结构分析

\subsubsection{篇章}

篇章衔接

篇章连贯

篇章标注

\subsection{应用角度}

中文分词

语料库建设

信息检索

问答系统

自动文摘

信息抽取

机器翻译

\section{单字繁转简}
给机器一个繁体字,让机器给出它的对应的简体字。

如果一个字不出现在对照表中,系统会报错:

解决方法:

1.完善对照表,添加上新的字。

2.忽略这次错误,改成不做处理,照抄,或其他方案。

\textbf{事实上繁体字转简体字也存在一对多的情况 : 松 丑}

\section{单字简转繁}
导致的结果:搜索到多个匹配项

解决方法:从匹配到的多个结果中选择最适合的结果

先分词,然后根据上下文判断。

\clearpage

\section*{导航}
\label{sec:navigation}

在这里,你可以添加一些导航链接,如链接到、\hyperref[subsec:sub]{子节}、\hyperref[sec:second]{第二节}等。

\end{document}
